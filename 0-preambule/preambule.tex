% ---- PACKAGES --------------------------------------------------------------------------------------------------------

% indispensable pour l'encodage des caractères
\usepackage[utf8]{inputenc}
\usepackage[T1]{fontenc}

%\usepackage[titlepage, pagenumber]{polytechnique}                  
                                            % pour faire des beaux documents avec la charte graphique de l'X

%\usepackage{lipsum}                         % utile our générer du faux texte
%\usepackage{url}                            % pour insérer des url

% packages de configuration des pages
%\usepackage[left=3.5cm,right=2cm,top=2cm,bottom=2.5cm]{geometry}
                                            %réglages des marges du document selon vos préférences ou celles de votre établissement
%\usepackage{multicol}

% packages graphiques
\usepackage{graphicx}                       % pour insérer images et pdf entre autres
%\usepackage{color}                          % pour rajouter des joulis couleurs
%\usepackage{photo}                          % pour rajouter des photos

% packages scientifiques
%\usepackage{amsmath}                        % extensions de l'ams pour les mathématiques
%\usepackage{amsfonts}
%\usepackage{amssymb}       
%\usepackage{amsthm}                         % pour rajouter des theoremes et tout ce qui va avec

\usepackage[french]{babel}                  % pour un document en français, attention à bien le mettre à la fin !

% packages à insérer uniquement après babel
%\usepackage{listings}                       % pour insérer du code source

%\usepackage{hyperref}                       % rend actif les liens, références croisées, toc…

% ---- MACROS ---------------------------------------------------------------------------------------------------------

% Renomme ces sections pour quelque chose de plus français
\addto{\captionsfrench}{\renewcommand{\abstractname}{Résumé}}
\addto{\captionsfrench}{\renewcommand{\contentsname}{Sommaire}} % a remettre en table des matieres si placee a la fin

%Commandes personnalisées
\newcommand\nom[2]{#1 \textsc{#2}}

% ---- METADATA --------------------------------------------------------------------------------------------------------

% titre et sous-titre pour package polytechnique
%\title[Titre plus court]{Titre complet} 
%\subtitle{Sous-titre}

\title{My document}
\author{\nom{Léo}{Guillon}}
\date{}

%\logo[headers]{images/logos/poly.jpg} % logo pour package polytechnique
